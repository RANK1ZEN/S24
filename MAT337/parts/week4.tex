\documentclass[../main.tex]{subfiles}

\begin{document}

\section{Week IV}

\begin{problem}[\S4.1 I]
    Suppose that $U$ is a linear transformation from $\mathbb{R}^n$ into $\mathbb{R}^m$ that is \textbf{isometric}, meaning that $\|U \mathbf{x}\| = \|\mathbf{x}\|$ for all $\mathbf{x} \in \mathbb{R}^n$.
    \begin{enumerate}[label=(\alph*)]
        \item Prove that $\langle U\mathbf{x}, U \mathbf{y} \rangle = \langle \mathbf{x}, \mathbf{y} \rangle$ for all $\mathbf{x}, \mathbf{y} \in \mathbb{R}^n$.
        \item If $\{ \mathbf{v}_1, \cdots, \mathbf{v}_k \}$ is an orthonormal set in $\mathbb{R}^n$, show that $\{ U \mathbf{v}_1, \cdots, U \mathbf{v}_k \}$ is also orthonormal.
    \end{enumerate}
\end{problem}
\begin{enumerate}[label=(\alph*)]
    \item Using the isometric property, we have the following.
        \begin{align*}
            \langle U\mathbf{x}, U\mathbf{y} \rangle
            &= \|U\mathbf{x}\| \|U\mathbf{y}\| \cos\theta \\
            &= \|\mathbf{x}\| \|\mathbf{y}\| \cos\theta \\
            &= \langle \mathbf{x}, \mathbf{y} \rangle
        \end{align*}
    \item \#TODO
\end{enumerate}

\section{4.1 J \#TODO}
\begin{problem}
    \begin{enumerate}[label=(\alph*)]
        \item Let $U$ be an isometric linear transformation of $\mathbb{R}^n$ onto itself.
            Show that the $n$ columns of the matrix of $U$ form an orthonormal basis for $\mathbb{R}^n$.
    \end{enumerate}
\end{problem}

\section{4.2 D}
\begin{problem}
    Let $\mathbf{x}_0 \in \mathbb{R}^n$ and $R > 0$.
    Prove that $\{ \mathbf{x} \in \mathbb{R}^n : \|\mathbf{x} - \mathbf{x}_0\| \le R \}$ is complete.
\end{problem}
\begin{proof}
    Fix a Cauchy sequence $(\mathbf{x}_k)_{k = 1}^\infty$ in $X$ (the set above).
    By completeness of $\mathbb{R}^n$, $(\mathbf{x}_k)$ converges to some $\mathbf{a} \in \mathbb{R}^n$.
    Next, consider the sequence $(\| \mathbf{x}_k - \mathbf{x}_0 \|)_{k = 1}^\infty$ with $\| \mathbf{x}_k - \mathbf{x}_0 \| \le R$ for all $k$.
    By continuity of the norm, $\lim_{k \to \infty} \| \mathbf{x}_k - \mathbf{x}_0 \| \le R$ implies $\| \mathbf{a} - \mathbf{x}_0 \| \le R$, implies $\mathbf{a} \in X$ as needed.
\end{proof}

\section{4.2 F}
\begin{problem}
    Let $\mathbf{v}_0 = (x_0, y_0)$ with $0 < x_0 < y_0$.
    Define for all $n \ge 0$,
    \[
        \mathbf{v}_{n + 1} = (x_{n + 1}, y_{n + 1}) = \left(\sqrt{x_n y_n}, \frac{x_n + y_n}{2}\right).
    \]
    \begin{enumerate}[label=(\alph*)]
        \item Show by induction that $0 < x_n < x_{n + 1} < y_{n + 1} < y_n$.
        \item Then estimate $y_{n + 1} - x_{n + 1}$ in terms of $y_n - x_n$.
    \end{enumerate}
\end{problem}
Fix $n \ge 0$.
Consider the base case $n = 0$.
We trivially have $0 < x_0 < y_1 < y_0$, thus it suffices to show $x_0 < x_1 < y_1$.
For the first equality, we have $x_0^2 < x_0 y_0$ then $x_0 < \sqrt{x_0 y_0}$.
For the second equality, we have 

\begin{align*}
    y_{n + 1} - x_{n + 1} &= \frac{x_n + y_n}{2} - \sqrt{x_n y_n} \\
                          &= (\sqrt{x} )
\end{align*}

%\section{4.3 H}
\begin{problem}[\S4.3 H]
    Show that a subset of $\mathbb{R}^n$ is complete\footnote{Every Cauchy sequence converges to a point in the set.}
    if and only if it is closed\footnote{Contains all its limit points. A point is a limit point if there is some sequence that converges to it.}.
\end{problem}

% \begin{problem}
%     Show that a subset of $\mathbb{R}^n$ is complete if and only if it is closed.
% \end{problem}
\begin{proof}
    Let $A$ be a subset of $\mathbb{R}^n$.
    ($\Rightarrow$) Suppose $A$ is complete.
    For each limit point $x$ and the sequence $(x_k)$ which approaches $x$, $(x_k)$ must also be Cauchy, which by completeness of $\mathbb{R}^n$ shows that $x \in A$.
    ($\Leftarrow$) Conversely, suppose $A$ is closed.
    Let $(x_k)$ be a Cauchy sequence in $A$, by completeness of $\mathbb{R}^n$, it converges to some point in $\mathbb{R}^n$.
    By the closed property, this point must also converge to a point in $A$.
\end{proof}

\begin{problem}[\S4.4 F]
    Let $(x_n)_{n = 1}^\infty$ be a sequence in a compact set $K \subseteq \mathbb{R}^n$ that is not convergent.
    Show that there are two subsequences of this sequence that are convergent to different limit points.
\end{problem}
\begin{proof}
    Since $K$ is compact, we apply Bolzano-Weierstrass to obtain a subsequence $(x_{n_i})_{i = 1}^\infty$ that converges to $a \in K$.
    Since $(x_n)$ does not converge to $a$, we have some positive $\varepsilon$ such that for every natural $N$ there exists $n \ge N$ such that $\|x_n - a\| \ge \varepsilon$.
    This forms another subsequence $(x_{n_j})$ such that $\|x_{n_j} - a\| \ge \varepsilon$ for all $j$.
    Apply Bolzano-Weierstrass again to $(x_{n_j})$ to obtain a subsequence that converges to a point different than $a$.
\end{proof}


\section{Extras}
\begin{problem}
    Suppose that $A$ is an $n \times n$ square matrix, and $\mathbf{v} \in \mathbf{R}^n$.
    Decide when the sequence $\mathbf{x}_k = A^k\mathbf{v}$ converges, and to what it converges to, based on the vector $\mathbf{v}$ and the eigenvalues and eigenvectors of $A$.
\end{problem}

% 3.2 B
Show that if $(|a_n|)_{n=1}^\infty$ is summable, then so is $(a_n)_{n=1}^\infty$.
\begin{proof}
    This is a case of the comparison test.
    $(|a_n|)$ is summable and $|a_n| \le |a_n|$ then $a_n$ is summable.
\end{proof}

\end{document}
