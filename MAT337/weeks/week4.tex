\documentclass[../main.tex]{subfiles}

\begin{document}

\section{Week IV}

\begin{problem}[\S4.3 H]
Show that a subset of $\mathbb{R}^n$ is complete
\footnote{Every Cauchy sequence converges to a point in the set.}
if and only if it is closed.
\footnote{Contains all its limit points.
A point is a limit point if there is some sequence that converges to it.}
\end{problem}

\begin{proof}
	Let $A$ be a subset of $\mathbb{R}^n$.
	($\Rightarrow$) Suppose $A$ is complete.
	For each limit point $x$ and the sequence $(x_k)$ which approaches $x$, $(x_k)$ must also be Cauchy, which by completeness of $\mathbb{R}^n$ shows that $x \in A$.

	($\Leftarrow$) Conversely, suppose $A$ is closed.
	Let $(x_k)$ be a Cauchy sequence in $A$, by completeness of $\mathbb{R}^n$, it converges to some point in $\mathbb{R}^n$.
	By the closed property, this point must also converge to a point in $A$.
\end{proof}

\begin{problem}[\S4.4 F]
Let $(x_n)_{n = 1}^\infty$ be a sequence in a compact set $K \subseteq \mathbb{R}^n$ that is not convergent.
Show that there are two subsequences of this sequence that are convergent to different limit points.
\end{problem}

\begin{proof}
	Since $K$ is compact, we apply Bolzano-Weierstrass to obtain a subsequence $(x_{n_i})_{i = 1}^\infty$ that converges to $a \in K$.
	Since $(x_n)$ does not converge to $a$, we have some positive $\varepsilon$ such that for every natural $N$ there exists $n \ge N$ such that $\|x_n - a\| \ge \varepsilon$.
	This forms another subsequence $(x_{n_j})$ such that $\|x_{n_j} - a\| \ge \varepsilon$ for all $j$.
	Apply Bolzano-Weierstrass again to $(x_{n_j})$ to obtain a subsequence that converges to a point different than $a$.
\end{proof}

\begin{problem}[\S4.4 I]
Let $A$ and $B$ be disjoint closed subsets of $\mathbb{R}^n$.
Define
\[
	d(A, B) = \inf\{\|a - b\| : a \in A, b \in B\}.
\]
\begin{itemize}
	\item[(a)] If $A = \{a\}$ is a singleton, show that $d(A, B) > 0$.
	\item[(b)] If $A$ is compact, show that $d(A, B) > 0$.
	\item[(c)] Find an example of two disjoint closed sets in $\mathbb{R}^2$ with $d(A, B) = 0$.
\end{itemize}
\end{problem}

For contradiction, assume that $d(A, B) = 0$.
Since $d$ is defined as an infimum, we can find some sequence $(\|a - b_n\|)_n$ with $\lim_{n \to \infty} (a - b_n) = 0$ where $a \in A$ is the single element and $b_n \in B$ is a sequence.
Thus the limit of $(b_n)_n$ considered as a sequence approaches $a \in A$.
Given that $B$ is closed, $a \in B$.

We can apply the same idea if $A$ is compact.
Assume the same contradiction, in this case, we have some sequence $(\|a_n - b_n\|)_n$ with limit $0$.
Thus $a_n$ approaches $x$ in $A$ and $b_n$ approaches the same $x$ but then $x$ must be in $B$.
Again, the intersection is nonempty.

Two closed sets is not strong enough of a condition.
Take for a counter example the subset where $y = 0$ and the graph of $\frac{1}{x}$.

\end{document}
