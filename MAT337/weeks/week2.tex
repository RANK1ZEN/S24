\documentclass[../main.tex]{subfiles}

\begin{document}

\section{Week II}

\begin{problem}
Let $a, b$ be positive real numbers.
Set $x_0 = a$ and $x_{n + 1} = (x_n^{-1} + b)^{-1}$ for $n \ge 0$.
\begin{enumerate}[label=(\alph*)]
	\item Prove that $x_n$ is monotone decreasing.
	\item Prove that the limit exists and find it.
\end{enumerate}
\end{problem}
\begin{enumerate}[label=(\alph*)]
	\item Notice that the sequence is positive, hence we have the following equality for all $n \ge 0$.
	      \begin{align*}
		      b x_n + 1            & > 1                   \\
		      x_n ((x_n)^{-1} + b) & > 1                   \\
		      x_n                  & > (x_n^{-1} + b)^{-1}
	      \end{align*}
	      The last equality implies $x_n > x_{n + 1}$.
	\item Since the sequence is positive, it has a lower bound 0 and by the Monotone convergence theorem, it converges.
	      Hence we know
	      \[
		      \lim_{n \to \infty} x_n = L
	      \]
	      exists for some $L$.
	      \begin{align*}
		      \lim_{n \to \infty} (x_n^{-1} + b)^{-1}
		        & = (L^{-1} + b)^{-1} \\
		      L & = (L^{-1} + b)^{-1}
	      \end{align*}
	      $L$ becomes 0.
\end{enumerate}

\begin{problem}[\S2.7 A]
Show that
\[
	(a_n) = \left( \frac{n \cos^n(n)}{\sqrt{n^2 + 2n}} \right)_{n = 1}^\infty
\]
has a convergent subsequence.
\end{problem}
By the Bolzano-Weierstrass Theorem, it simply suffices to show that $(a_n)$ is bounded.
This can be readily shown since for all $n \ge 1$, $\cos^n(n) \in [-1, 1]$ and $\frac{n}{\sqrt{n^2 + 2n}} < 1$.

\begin{problem}[\S2.7 I]
Suppose $(x_n)_{n = 1}^\infty$ is a sequence in $\mathbb{R}$, and that $L_k$ are real numbers with $\lim_{k \to \infty} L_k = L$.
If for each $k \ge 1$, there is a subsequence of $(x_n)_{n = 1}^\infty$ converging to $L_k$, show that some subsequence converges to $L$.

\textbf{HINT}: Find an increasing sequence $n_k$ such that $|x_{n_k} - L| < 1/k$.
\end{problem}
Construct a sequence $(n_k)_{k = 1}^\infty$ by the following procedure for each $k$.
Consider the fact that the sequence $(L_j)_{j = 1}^\infty$ approaches $L$.
We can require some $J$ such that $|L_{J} - L| < \frac{1}{2k}$.
For this $J$, we are given a subsequence $(x_{s_{i}})_{i = 1}^\infty$ that converges to $L_J$, and thus we can require $I \in \mathbb{N}$ such that for $i \ge I$, $|x_{s_i} - L_J| < \frac{1}{2k}$.
Take $n_k$ as some $s_i$ where $i \ge I$ and $n_k$ is greater than all the previous terms.
Hence $(n_k)$ satisfies an increasing sequence where for each $k$,
\[
	|x_{n_k} - L| \le |x_{n_k} - L_J| + |L_J - L| < \frac{1}{k}.
\]
Since we have $|x_{n_k} - L| < \frac{1}{k}$ for every $k$, we can conclude that $x_{n_k}$ converges to $L$.

\begin{problem}[\S2.7 J]
(\textbf{a}) Suppose that $(x_n)_{n = 1}^\infty$ is a sequence of real numbers.
If $L = \liminf x_n$,\footnote{$\liminf$ is defined as
	\[ \liminf x_n := \lim_{n \to \infty} \left(\inf_{m \ge n} x_m\right) \]}
show that there is a subsequence $(x_{n_k})_{k = 1}^\infty$ such that $\lim_{k \to \infty} x_{n_k} = L$.
\end{problem}
{\huge TODO}

\begin{problem}[\S2.8 A]
Let $(x_n)$ be Cauchy with a subsequence $(x_{n_k})$ such that
\[
	\lim_{k \to \infty} x_{n_k} = a.
\]
Show that $\lim_{n \to \infty} x_n = a$.
\end{problem}
Fix $\varepsilon > 0$.
We will the limit definitions of $(x_n)$ and $(x_{n_k})$ to require that:
\begin{itemize}
	\item There exists $M$ such that for every $n, m \ge M$, $|x_n - x_m| < \varepsilon/2$.
	\item There exists $K$ such that for every $k \ge K$, $|x_{n_k} - a| < \varepsilon/2$.
\end{itemize}
Take $N = \max\{ M, K \}$, so that we have both conditions.
We have that for $n \ge N$,
\begin{align*}
	|x_n - a|
	 & = |x_n - x_{n_k} + x_{n_k} - a|               \\
	 & \le |x_n - x_{n_k}| + |x_{n_k} - a|           \\
	 & < \varepsilon/2 + \varepsilon/2 = \varepsilon
\end{align*}

\begin{problem}[\S3.1 C]
Prove that if $p > 1$ and $\sum_{k = 1}^\infty t_k$ is a convergent series of nonnegative numbers, $\sum_{k = 1}^\infty t_k^p$ converges.
\end{problem}
Proceed by the Cauchy criterion for series.
Fix $\varepsilon > 0$.
Without loss, assume $\varepsilon < 1$.
Since $t_k$ is convergent, there is an $N$ such that for all $n, m \ge N$,
\[
	\sum_{k = n + 1}^m t_k < \varepsilon
\]
Since $\varepsilon < 1$, $t_k < 1$ for $n < k \le m$.
Hence $t_k^p < t_k$ for $n < k \le m$.
Thus we have

\begin{problem}[\S3.1 D]
Let $(a_n)_{n = 1}^\infty$ be a sequence such that $\lim_{n \to \infty} |a_n| = 0$.
Prove that there is a subsequence $(a_{n_k})$ such that $\sum_{k = 1}^\infty a_{n_k}$ converges.
\end{problem}
For each $k \ge 1$, we will take $n_k$ as

$N$ such that for $n \ge N$, $|a_{n}| < 1/k^2$.

\begin{problem}[\S3.2 H]
Show that if $\sum_{n = 1}^\infty a_n$ and $\sum_{n = 1}^\infty b_n$ are series with $b_n \ge 0$ such that $\limsup_{n \to \infty} \frac{|a_n|}{b_n} < \infty$ and $\sum_{n = 1}^\infty b_n < \infty$, then the series $\sum_{n = 1}^\infty a_n$ converges.
\end{problem}

\begin{proof}
	Since the limit superior of $\frac{|a_n|}{b_n}$ is a real number, the sequence is bounded above.
	Suppose the sequence is bounded by $M$, then for every $n$, $\frac{|a_n|}{b_n} < M$, and thus the sequence $|a_n|$ is bounded by the sequence $M b_n$.
	By the Comparison Test,
	\footnote{For sequences $(a_n), (b_n)$ with $|a_n| \le b_n$, if $(b_n)$ is summable, then so is $(a_n)$.}
	since $b_n$ converges, so must $a_n$.
\end{proof}

\end{document}
