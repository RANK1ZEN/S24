\documentclass[../main.tex]{subfiles}

\begin{document}

\section{Week II - The Axiom of Choice and its relatives, order topologies, finite products}

\begin{problem}[1]
    Let $X$ be a set:
    \begin{enumerate}[label=(\alph*)]
        \item Prove that there is no surjective function $f : X \to \mathbf{P}(X)$.
        \item Conclude that uncountable sets exists.
    \end{enumerate}
\end{problem}
\begin{enumerate}[label=(\alph*)]
    \item Let $D = \{ x \in X : x \notin f(x) \}$.
        Fix $a \in X$, if $a \notin D$, then $a \in f(a)$ hence $D \ne f(a)$.
        If $a \in D$, then $a \notin f(a)$ hence $D \ne f(a)$.
        This shows that no element in $X$ maps to $D$.
    \item There is no surjection from $\mathbb{N}$ to $\mathbf{P}(\mathbb{N})$ hence $\mathbf{P}(\mathbb{N})$ is not countable.
\end{enumerate}

\begin{problem}[2]
    Let $\{ A_n : n \in \mathbb{N} \}$ be a countable collection of countable sets:
    \begin{enumerate}[label=(\alph*)]
        \item Prove that their union
            \[ \bigcup_{n \in \mathbb{N}} A_n \]
            is countable as well.
        \item Prove that the product $A_1 \times A_2$ is countable.
        \item Conclude that for every $n \in \mathbb{N}$, the finite product
            \[ \prod_{k \le n} A_k \]
            is countable.
    \end{enumerate}
\end{problem}
\begin{enumerate}[label=(\alph*)]
    \item It suffices to show a surjection from $\mathbb{N} \times \mathbb{N}$ to $\bigcup_{n \in \mathbb{N}} A_n$ since we know $\mathbb{N} \times \mathbb{N}$ is countable.
        For each $n \in \mathbb{N}$, let $f_n: \mathbb{N} \to A_n$ be the surjection from $\mathbb{N}$ to $A_n$.
        Define a new function
        \[
            g : \mathbb{N} \times \mathbb{N} \to \bigcup_{n \in \mathbb{N}} A_n \text{ given by } g(a, b) = f_a(b) \; ,
        \]
        for $a, b \in \mathbb{N}$.
        $g$ is indeed a surjection onto $\bigcup_{n \in \mathbb{N}} A_n$ since $f_n$ is surjection.
    \item Let $f_1, f_2$ be the respective surjections.
        Consider a natural number of the form $2^q 3^p$, and map this number to $(f_1(q), f_2(p))$ if the functions are defined at their respective values.
        This is a surjection from $\mathbb{N}$ to $A_1 \times A_2$.
    \item We can use the same method, just take a list of the first $n$ primes.
\end{enumerate}

\begin{problem}[3]
    Prove that the set $\{ 0, 1 \}^{\mathbb{N}}$ of functions $f : \mathbb{N} \to \{ 0, 1 \}$ is infinite and uncountable.
\end{problem}
Construct a bijection $g$ between $\{ 0, 1 \}^{\mathbb{N}}$ and $\mathbf{P}(\mathbb{N})$ given by the following.
\[
    g(f) = \{ n \in \mathbb{N} : f(n) = 1 \} \text{ for } f \in \{ 0, 1 \}^{\mathbb{N}}
\]
It is clear $g$ is bijective, hence since $\mathbf{P}(\mathbb{N})$ is infinite and uncountable, $\{ 0, 1 \}^{\mathbb{N}}$ is also infinite and uncountable.

\begin{problem}[5]
    Let $(P, \le)$ be a nonempty partially ordered set such that every chain in $P$ is finite.
    Prove that $P$ has a maximal element.
    Can you sketch a similar proof for Zorn's Lemma?
\end{problem}
For contradiction, suppose that $P$ does not have a maximal element, i.e., for every element in $P$, there is another element strictly greater than it.
Using this assumption, we can construct a chain in $P$ is not finite.
First fix $a_1 \in P$.
Then we can find $a_2 \in P$ such that $a_1 \preceq a_2$.
In general, given a chain
\[
    a_1 \preceq a_2 \preceq \cdots \preceq a_n \; ,
\]
we can find an $a_{n + 1}$ such that $a_n \preceq a_{n + 1}$ and $\{ a_1, \cdots ,a_{n + 1} \}$ is another chain in $P$.
Contradiction is met by constructing an infinite chain.

\begin{problem}[6]
    Prove the Axiom of Choice by assuming the Well-ordering Principle.

    \textbf{Hint:} Be very careful on how do you choose your well-ordering.
\end{problem}
Fix a collection of nonempty sets $\mathcal{A} := \{ A_\alpha : \alpha \in \Lambda \}$.
We can define a set
\[
    W := \bigcup_{\alpha \in \Lambda} A_\alpha \; ,
\]
and choose a well-ordering.
By definition of the Cartesian product, we require that the following set is nonempty.
\[
    \prod_{\alpha \in \Lambda} A_\alpha = \left\{ \left( f : \Lambda \to W \right) : \forall \alpha \in \Lambda, f(\alpha) \in A_\alpha \right\}
\]
We will construct $f : \Lambda \to W $ by taking $f(\alpha) = \min\{ a \in A : a \in A_\alpha \}$.
Note that the least element exists because $A_\alpha$ is a nonempty subset of $W$ with a well ordering.

\begin{problem}[7]
    Given a set $X$, let
    \[
        P_X := \{ (A, \le_A) : A \subseteq X \text{ and } \le_A \text{ is a well-ordering of } A \}
    \]
    \begin{enumerate}[label=(\alph*)]
        \item Prove that P is nonempty.
        \item Let $\preceq$ be the relation on $P$ given by $(A, \le_A) \preceq (B, \le_B)$ if, and only if,
            \[
                A \subseteq B \text{ and } \le_B \textit{ extends } \le_A \; ,
            \]
            i.e., if $x, y \in A$ and $x \le_A y$ then $x \le_B y$.
            Prove that $(P, \preceq)$ is a partially ordered set.
        \item Prove that every chain in $P$ is bounded.
        \item Conclude that there exists a well-ordering of $X$.
    \end{enumerate}
\end{problem}
\begin{enumerate}[label=(\alph*)]
    \item Every set is well-orderable, thus every subset of $X$ has a well-ordering of $A$.
    \item Fix $(A, \le_A), (B, \le_B), (C, \le_C) \in P$.

        (1) Indeed, $(A, \le_A) \preceq (A, \le_A)$ since $A \subseteq A$ and $\le_A$ extends $\le_A$.

        (2) Assume that $(A, \le_A) \preceq (B, \le_B)$ and $(B, \le_B) \preceq (A, \le_A)$.
        Then we have that $A \subseteq B$ and $B \subseteq A$ as well as $\le_B$ extends $\le_A$ and $\le_A$ extends $\le_B$.
        Thus $A = B$ and $\le_A = \le_B$.

        (3) Assume that $(A, \le_A) \preceq (B, \le_B)$ and $(B, \le_B) \preceq (C, \le_C)$.
        Then we have $A \subseteq B \subseteq C$ and $\le_A$ extends $\le_B$ extends $\le_C$.
        Thus $(A, \le_A) \preceq (C, \le_C)$.

        $P$ is a partial order since (1), (2), (3) are satisfied.
    \item For contradiction, let $C$ be a chain in $P$ that is not bounded, i.e.,
        for every $p \in C$, there exists $q \in C$ such that $p \prec q$.
        $p \le B$
    \item Since $P$ is a nonempty partially ordered set such that every chain is bounded, we can conclude by Zorn's lemma that $P$ has a maximal element.
\end{enumerate}

\begin{problem}[8]
    Recall that the distance between two real numbers is defined as the function $d: \mathbb{R} \times \mathbb{R} \to \mathbb{R}$ given by $d\langle x, y \rangle = |x - y|$.
    We will say that a subset $A \subseteq \mathbb{R}$ is \textit{distance-injective} if for all $x, y, z, w \in A$, if $d\langle x, y \rangle = d\langle z, w \rangle > 0$, then $\{ x, y \} = \{z, w \}$.

    Prove that there exists an uncountable, distance-injective subset of $\mathbb{R}$.
\end{problem}
Let $A_1$ 

\begin{problem}[9]
    Let $(X, \le)$ be a well-ordered set:
    \begin{enumerate}[label=(\alph*)]
        \item Prove that every non-maximal element of $X$ has an immediate successor.
        \item Prove that every bounded subset of $X$ has a least upper bound.
        \item Prove that $\omega_1$ has no largest element.
        \item Prove that the set of $\alpha \in \omega_1$ with no immediate predecessor is uncountable.
        \item Conclude that $\omega_1$ endowed with the order topology is not a discrete space.
    \end{enumerate}
\end{problem}
\begin{enumerate}[label=(\alph*)]
    \item Fix a non-maximal element $x \in X$ and consider the set elements larger than it
        \[
            A := \{ y \in X : x \lneq y \}
        \]
        $A$ is nonempty since $x$ is non-maximal, thus $A$ has a least element, which is the immediate successor of $x$.
    \item Fix a bounded subset $Q \subseteq X$ and consider the set of all upper bounds:
        \[
            A := \{ y \in X : q \le y \text{ for all } q \in Q \}
        \]
        $A$ is nonempty since $Q$ being bounded implies at least one bound.
        Since $A$ is nonempty, it has a least element, which is the least upper bound.
    \item hi
\end{enumerate}


\begin{problem}[10]
    Let $X$ and $Y$ be topological spaces.
    A function $f : X \to Y$ is called \textit{open} if the set
    \[
        f[U] := \{ f(x) : x \in U \}
    \]
    is an open subset of $Y$ for every open subset $U$ of $X$.

    Prove that the projections $\pi_1 : X \times Y \to X$ and $\pi_2 : X \times Y \to Y$ are open.
\end{problem}
Fix an open set $U \in X \times Y$.
\[
    U = \bigcup
\]
Let $U \times V$ where $U$ is an open subset of $X$ and $V$ is an open subset of $Y$.


\end{document}
