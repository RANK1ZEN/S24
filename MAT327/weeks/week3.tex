\documentclass[../main.tex]{subfiles}

\begin{document}

\section{Week III - Subspaces, closed sets and limit points, Hausdorff spaces}

\begin{problem}
    Prove that the topology of the lexicographic plane is the same as the product topology $\mathbb{R}_d \times \mathbb{R}$.
\end{problem}
\begin{proof}
    It suffices to consider the basic open sets in both topologies.
    Every open interval in the lexicographic plane can be expressed as a union of open sets in $\mathbb{R}_d \times \mathbb{R}$.
    Hence the topology on $\mathbb{R}_d \times \mathbb{R}$ is finer than the dictionary topology.

    Conversely, open basic sets in $\mathbb{R}_d \times \mathbb{R}$ take the form $\{ x \} \times (y_1, y_2)$, which are themselves open intervals in the lexicographic plane.
    Hence these topologies are the same.
\end{proof}

\begin{problem}
    Prove that if $Y$ is a convex
    subset of an ordered set $X$, then the order topology on $Y$ is equal to the topology on $Y$ as a subspace of $X$ endowed with the order topology.
\end{problem}
\sidenote{A subset $Y$ of an ordered set $X$ is called \textit{convex} if for all $x, y \in Y$, the interval $[x, y]$ is included in $Y$ as well.}
\begin{proof}
    ($\subseteq$) Consider the order topology on $Y$, and open rays in $Y$ which are a subbasis for the order topology.
    Open rays in the order topology are simply intersections of open rays in $X$ with $Y$, in particular, they are also open in the subspace topology. 
    Since open rays form a subbasis
    \sidenote{If $\tau_1$ and $\tau_2$ are topologies on a set, showing $S_1 \subseteq \tau_2$ or $\mathscr{B} \subseteq \tau_2$ shows that $\tau_1 \subseteq \tau_2$.},
    the order topology is included in the subspace topology.

    ($\supseteq$) Consider the subspace topology on $Y$, and a subbasis consisting of intersections $Y$ and open rays of $X$
    Since $Y$ is convex, these intersections are still open rays in $Y$, which shows that the subspace topology in included in the order topology.
\end{proof}

\begin{problem}
    If $L$ is a straight line in the plane, describe the topology $L$ inherits as a subspace of (a) $\mathbb{R}_\ell \times \mathbb{R}$ and as a subspace of (b) $\mathbb{R}_\ell \times \mathbb{R}_\ell$.
    In each case it is a familiar topology.
\end{problem}
\begin{enumerate}[label=(\alph*)]
    \item The general basis element of $\mathbb{R}_\ell \times \mathbb{R}$ is
        \[
            [a, b) \times (c, d),
        \]
        for real numbers $a, b, c, d$.
        The general basis element that $L$ inherits as a subspace is
        \[[a, b) \times (c, d) \cap L.\]
\end{enumerate}

\begin{problem}
    A collection $\mathscr{C}$ of subsets of a set $X$ is called a \emph{closed basis for a topology on $X$} if:
    \begin{enumerate}[label=(\roman*)]
        \item The intersection of $\mathscr{C}$ is empty.
        \item For all $C_1, C_2 \in \mathscr{C}$, and all $x \notin C_1 \cup C_2$, there is some $C_3 \in \mathscr{C}$ such that
            \[
                x \notin C_3 \text{ and } C_1 \cup C_2 \subseteq C_3.
            \]
    \end{enumerate}

    \begin{enumerate}
        \item[(a)] Prove that a collection $\mathscr{C}$ of subsets of $X$ is a closed basis for a topology on $X$ if, and only if, the collection $\{U\subseteq X:X\setminus U\in\mathscr{C}\}$ is a basis for a topology on $X$.
    \end{enumerate} 

    Let now $\mathbb{C}$ be the set of complex numbers and let $n\in\mathbb{N}$. For each set $S$ of polynomials with complex coefficients and $n$-many variables, let
    \[
        V(S):=\left\{\vec x\in\mathbb{C}^n:(\forall f\in S)(f\left(\vec x\right)=0)\right\}.
    \]
    \begin{enumerate}
        \item[(b)] Prove that the collection
            $\{V(S): S$ is a set of polynomials with complex coefficients and $n$-many variables$\}$
            is a closed basis for a topology on $\mathbb{C}^n$.
        \item[(c)] Can you describe this topology explicitly when $n=1$?

        \textbf{Hint:} It is a familiar topology that we already defined in class.
    \end{enumerate}
\end{problem}
\begin{enumerate}[label=(\alph*)]
    \item Fix a collection of subsets $\mathscr{C}$ and define $\mathscr{B} := \{U\subseteq X:X\setminus U\in\mathscr{C}\}$.
        To show one side of the implication, assume $\mathscr{C}$ be a closed basis.
        \begin{enumerate}
            \item[(1)] Fix $x \in X$.
                Since the intersection of $\mathscr{C}$ is empty, there exists a $C \in \mathscr{C}$ such that $x \notin C$.
                Then $C^\complement \in \mathscr{B}$ with $x \in C^\complement$.
            \item[(2)] Fix two sets $B_1, B_2 \in \mathscr{B}$.
                These sets have the property that ${B_1}^\complement \in \mathscr{C}$ and ${B_2}^\complement \in \mathscr{C}$.
                For elements in the intersection $x \in B_1 \cap B_2$, we have that $x \notin {B_1}^\complement \cap {B_2}^\complement$.
                By the second property of the closed basis, we have some $C_3 \in \mathscr{C}$ such that $x \notin C_3$ and ${B_1}^\complement \cap {B_2}^\complement \subseteq C_3$.
                $C_3$ satisfies ${C_3}^\complement \in \mathscr{B}$, which satisfy $x \in {C_3}^\complement$ and ${C_3}^\complement \subseteq B_1 \cup B_2$.
        \end{enumerate}
        To show the other side of the implication, assume $\mathscr{B}$ is a basis.
        \begin{enumerate}[label=(\roman*)]
            \item Fix $x \in X$. % FIXME
                There is a basis $B$ such that $x \in B$ and hence $B^\complement \in $
            \item Fix $C_1, C_2 \in \mathscr{C}$ and $x \notin C_1 \cup C_2$.
                Since ${C_1}^\complement, {C_2}^\complement \in \mathscr{B}$ and $x \in {C_1}^\complement \cap {C_2}^\complement$, there exists $B_3 \in \mathscr{B}$ such that $x \in B_3 \subseteq {C_1}^\complement \cap {C_2}^\complement$.
                Thus $x \notin {B_3}^\complement$ and $C_1 \cup C_2 \subseteq {B_3}^\complement$.
        \end{enumerate}
    \item (i), it is kinda obvious that no vectors satisfy the whole thing.
\end{enumerate}

\begin{problem}
    Let $A$ and $B$ be subsets of a topological space $X$.
    Either prove or give a counterexample for the following equalities:
    \begin{enumerate}[label=(\alph*)]
        \item $\overline{A\cup B}=\overline{A}\cup\overline{B}$
        \item $\overline{A\cap B}=\overline{A}\cap\overline{B}$
        \item $\overline{A\setminus B}=\overline{A}\setminus\overline{B}$
        \item $\overline{A\times B}=\overline{A}\times\overline{B}$
    \end{enumerate}
\end{problem}
\begin{enumerate}[label=(\alph*)]
    \item ($\supseteq$)
        We have $A, B \subseteq A \cup B$ hence $\overline{A}, \overline{B} \subseteq \overline{A \cup B}$ and their unions are also contained.

        ($\subseteq$)
        Fix a point $x \in \overline{A \cup B}$.
        By definition, $x$ belongs to every closed set containing $A \cup B$.
        Notice that $\overline{A} \cup \overline{B}$ is a closed set containing $A \cup B$, hence $x \in \overline{A} \cup \overline{B}$.
    \item This is false.
        Take $\mathbb{Q}$ and $\mathbb{R} \setminus \mathbb{Q}$ as subsets of $\mathbb{R}$ with the usual topology.
        Consider the respective closures both $\mathbb{R}$, which has intersection $\mathbb{R}$.
        But $\mathbb{R} \cap \mathbb{R} \setminus \mathbb{Q} = \emptyset$ is empty, of which taking the closure is still empty.
    \item H
\end{enumerate}

\begin{problem}
    Let $A$ be a subset of a topological space $X$:
    \begin{enumerate}[label=(\alph*)]
        \item Prove that $A$ is open if, and only if, $A = A^\circ$.
        \item Prove that $A$ is closed if, and only if, $A = \overline{A}$.
    \end{enumerate}
\end{problem}
\begin{enumerate}[label=(\alph*)]
    \item Assume $A$ is open, it suffices to show $A \subseteq A^\circ$, since the other inclusion is trivial.
        Fix $x \in A$, we can find a basis 
        Conversely, assume $A = A^\circ$, then $A$ is a union of open sets, which is open.
    \item Assume $A$ is closed, it suffices to show $\overline{A} \subseteq A$.
        Fix $x \in \overline{A}$, % FIXME
        Conversely, assume $A = \overline{A}$, $A$ is a intersection of closed sets, which is closed.
\end{enumerate}

\begin{problem}
    Compute the interior and the closure of the following sets:
    \begin{enumerate}[label=(\alph*)]
        \item $(0, 1]$ as a subset of the Sorgenfrey line.
    \end{enumerate}
\end{problem}
\begin{enumerate}[label=(\alph*)]
    \item Closure $a$
\end{enumerate}

\begin{problem}
\end{problem}

\begin{problem}
    Prove the following statements:
    \begin{enumerate}[label=(\alph*)]
        \item Every order topology is Hausdorff.
        \item The product of two Hausdorff spaces is Hausdorff.
        \item A subspace of a Hausdorff space is Hausdorff.
    \end{enumerate}
\end{problem}
\begin{enumerate}[label=(\alph*)]
    \item For any two points $x, y$ in an order topology, we can find some basic interval around $x$, then find another interval around $y$ that doesn't intersect with the first interval.\sidenote{This is omega hand-wavy.}
    \item Fix two points $(x_1, x_2), (y_1, y_2)$ in the product of Hausdorff spaces.
\end{enumerate}

\begin{problem}
\end{problem}

\end{document}
