\documentclass[../main.tex]{subfiles}

\begin{document}

\section{Week I - Introduction, topologies and topological spaces, bases and subbases}

\begin{problem}[1]
    Let $X$ be a nonempty set and fix an element $x \in X$.
    Prove that the set
    \[ \tau_x := \{ U \subseteq X : x \in U \} \cup \{ \emptyset \} \]
    is a topology on $X$.
\end{problem}
\begin{itemize}
    \item By definition $X$, $\emptyset \in \tau_x$.
    \item Let $U, W \in \tau_x$ with $x \in U$ and $x \in W$.
        We can conclude that $x \in U \cap W$ hence $U \cap W \in \tau_x$.
    \item To show arbitrary union, let $U_{\alpha} \in \tau_x$ for $\alpha \in \Lambda$.
        The union of all $U_{\alpha}$ still contains $x$,
        which means the union of $U_{\alpha}$ is in $\tau_x$.
\end{itemize}

\begin{problem}[2]
    Let $X$ be a set and let $(Y, \tau)$ be a topological space.
    Prove that the set
    \[ \tau_f := \{ f^{-1}(V) : V \in \tau \} \]
    is a topology on $X$.
\end{problem}
\begin{itemize}
    \item Since $\tau$ is a topology, $f^{-1}(Y), f^{-1}(\emptyset) \in \tau_f$.
        By definition of the preimages, $X, \emptyset \in \tau_f$, respectively.
    \item Let $f^{-1}(V_\alpha)$ for $\alpha \in \Lambda$ be some arbitrary sets in $\tau_f$.
        Using a property of the preimage,
        \[
            \bigcup_{\alpha \in \Lambda} f^{-1}(V_\alpha) = f^{-1}\left(\bigcup_{\alpha \in \Lambda} V_\alpha\right)
        \]
        which is in $\tau_f$ since $\bigcup_{\alpha \in \Lambda} V_\alpha \in \tau$.
    \item Similary, for sets $f^{-1}(U), f^{-1}(V) \in \tau_f$,
        \[
            f^{-1}(U) \cap f^{-1}(V) = f^{-1}(U \cap V)
        \]
        which is in $\tau_f$ since $U \cap V \in \tau$.
\end{itemize}

\begin{problem}[3]
    Prove that if $\{ \tau_\alpha: \alpha \in \Lambda \}$ is a collection of topologies on a set $X$,
    then their intersection
    \[ \bigcap_{\alpha \in \Lambda} \tau_\alpha \]
    is a topology as well.
    Is the union of topologies a topology?
\end{problem}
Since $X, \emptyset$ is in every topology, $X, \emptyset$ will also be in the intersection of the topologies.
Fix some sets in $\bigcap_{\alpha \in \Lambda} \tau_\alpha$.
Fix an $\alpha$, since these sets belong to $\tau_\alpha$, the union will also be in $\tau_\alpha$.
Since $\alpha$ is arbitrary, the union will belong in $\tau_\alpha$ for every $\alpha \in \Lambda$.
Thus the union of sets is in $\bigcap_{\alpha \in \Lambda} \tau_\alpha$.
The exact argument works for finite sets and their intersections.

However, the union of topologies is not a topology.

\begin{problem}[4]
    Prove that the collection $\mathscr{B} := \{ (p, q) : p, q \in \mathbb{Q} \}$ is a basis for the standard topology on $\mathbb{R}$.
\end{problem}
VERIFY THAT $\mathscr{B}$ is indeed a basis.
Let $x$
$\mathscr{B}$ is clearly a basis since it covers everything and a ...
Let $\tau$ be the topology generated by basis $\mathscr{B}$.
$\tau \subseteq $
To show that a basis generates $\tau$, we want to check that every element of $\tau$ is a union of elements of

\begin{problem}[5]
    Let $(X, \tau)$ be a topological space.
    A collection of nonempty open sets $\mathscr{P}$ is called a $\pi$-basis for $\tau$ if for every nonempty open set $U$ there exists some $P \in \mathscr{P}$ such that $P \subseteq U$.
    \begin{enumerate}[label=(\alph*)]
        \item Prove that if $\mathscr{B}$ is a basis for a topology $\tau$, then $\mathscr{B} \setminus \{ \emptyset \}$ is a $\pi$-basis for $\tau$.
        \item Prove that the collection
            \[
                \mathscr{P} := \{ [p, q) : p, q \in \mathbb{Q} \text{ and } p < q \}
            \]
            is a $\pi$-basis, but is \textbf{not} a basis for the lower limit topology.
    \end{enumerate}
\end{problem}
\begin{enumerate}[label=(\alph*)]
    \item Fix a nonempty open set $U$ in $\tau$ and fix $x \in U$.
        There exists $B \in \mathscr{B}$ such that $x \in B \subseteq U$.
        $B$ is nonempty thus $B \in \mathscr{B} \setminus \{ \emptyset \}$ with $B \subseteq U$.
        Thus $\mathscr{B} \setminus \{ \emptyset \}$ is a $\pi$-basis.
        We also verify that $\mathscr{B} \setminus \{ \emptyset \}$ is a collection of nonempty sets.
    \item Fix a nonempty open set $U$ in the lower limit topology.
        Fix $x \in U$, we know that there is some $\varepsilon > 0$ such that $[x, x + \varepsilon) \subseteq U$.
        % FIXME: LOL
        So we can definitely find some $p, q$ as well.

        However, $\mathscr{P}$ is not a basis.
        For example, consider the lower endpoint of $[\pi, 6)$.
        At the point $\pi$, we are unable to find a set in $P \in \mathscr{P}$ such that $\pi \in P \subseteq [\pi, 6)$.
\end{enumerate}

\begin{problem}[6]
    Let $X$ be a nonempty set, and let $\mathbb{R}^X$ be the set of functions $f : X \to \mathbb{R}$.
    Given $A \subseteq X$, let
    \[
        U_A := \{ f \in \mathbb{R}^X : f(x) = 0 \text{ for all } x \in A \}
    \]
    Prove that the collection $\mathscr{B} := \{ U_A : A \subseteq X \}$ is a basis for a topology on $\mathbb{R}^X$.
\end{problem}
To show $\mathscr{B}$ covers $\mathbb{R}^X$,
fix $f \in \mathbb{R}^X$ and consider the preimage $f^{-1}(\{ 0 \})$.
We can construct the set
\[
    U_{f^{-1}(\{ 0 \})} = \{ g \in \mathbb{R}^X : g(x) = 0 \text{ for all } x \in f^{-1}(\{ 0 \}) \}
\]
so that $f \in U_{f^{-1}(\{ 0 \})}$.
To show property of intersections of basis,
suppose two basis elements $U_V, U_W \in \mathscr{B}$ with $V, W \subseteq X$.
Fix $f \in U_V \cap U_W$ and consider $U_{V \cup W}$.
We have $f \in U_{V \cup W} \subseteq U_V \cap U_W$.

\begin{problem}[7]
    Let $\mathscr{B}$ be a basis for a topology on a set $X$.
    Prove that the topology generated by $\mathscr{B}$ equals
    \[ \bigcap \{ \tau : \tau \text{ is a topology on } X \text{ and } \mathscr{B} \subseteq \tau \} \]
\end{problem}
First lets call this set $A$ and note that the intersection of topologies is still a topology.
Thus $A$ is a topology containing $\mathscr{B}$ and notably also contains unions of the basis elements.
Next let $T$ be the topology generated by $\mathscr{B}$ and fix $U \in T$.
One criteria is that $U$ is a union of basis elements, thus $U \in A$.
This shows $T \subseteq A$.

Conversely, it is trivial that $A \subseteq T$ since $T$ itself is a topology on $X$ that containes $\mathscr{B}$.

\begin{problem}[10]
    Given $a, b \in \mathbb{Z}$ , with $a \ne 0$, let
    \[
        S(a, b) := \{ ax + b : x \in \mathbb{Z} \}.
    \]
    \begin{enumerate}[label=(\alph*)]
        \item Prove that the collection
            \[
                \{ S(a, b) : a, b \in \mathbb{Z} \text{ and } a \ne 0 \}
            \]
            is a basis for a topology $\tau$ on $\mathbb{Z}$.
        \item Prove that every nonempty open set in the topology $\tau$ is infinite.
        \item Prove that for each $a, b \in \mathbb{Z}$, with $a \ne 0$, $\mathbb{Z} \setminus S(a, b)$ is open.
        \item Use the Fundamental Theorem of Arithmetic to prove that
            \[ \bigcap_{p \text{ prime}} \mathbb{Z} \setminus S(p, 0) = \{ -1, 1 \} \]
        \item Conclude that there are infinitely many prime numbers.
    \end{enumerate}
\end{problem}
\begin{enumerate}[label=(\alph*)]
    \item To show that every integer belongs some $S(a, b)$ in this collection,
        fix an arbitrary integer $x$.
        Consider the case when $x = 0$, notice that $0$ belongs to every $S(a, b)$ given $a$ divides $b$.
        In the case when $x \ne 0$, it is given that $x = a + b$ exists.
    \item Let $U \in \tau$, be a nonempty set in the topology.
        $U$ satisfies that for every $x \in U$, there exists $S(a, b)$ such that $x \in S(a, b) \subseteq U$.
        Since $S(a, b)$ is infinite, and $U$ contains $S(a, b)$, $U$ must also be infinite.
        Hence given that $U$ is nonempty, it is infinite.
    \item Fix $a, b \in \mathbb{Z}$ with $a \ne 0$.
        Let $x \in U \subseteq \mathbb{Z} \setminus S(a, b)$
        It holds that $x \ne ax + b$ for any $x$, but it means that $x = ax + y$ for some $x$ and $y$.
        Thus $x \in S(a, y) \subseteq U$.
    \item By De Morgan's Law, we obtain the following set.
        \[
            \bigcap_{p \text{ prime}} \mathbb{Z} \setminus S(p, 0)
            = \left( \bigcup_{p \text{ prime}} S(p, 0) \right)^C
        \]
        Thus it follows from the Fundamental Theorem of Arithmetic that any integer greater than 1 belongs inside of the complement, and is therefore not in the set.
        0 is also not in the set since $0 \in S(p, 0)$ is true for all primes.
        Hence the set is equal to $\{ -1, 1\}$.
\end{enumerate}

\end{document}
