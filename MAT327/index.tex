\documentclass[../main.tex]{subfiles}

\begin{document}

\section{Products}

\begin{problem}[Theorem 19.6]
    Let $f : A \to \prod_{\alpha \in J} X_\alpha$ be given by the equation
    \begin{equation}
        f(a) = (f_\alpha(a))_{\alpha \in J},
    \end{equation}
    where $f_\alpha : A \to X_\alpha$ for each $\alpha$.
    Let $\prod X_\alpha$ have the product topology.
    Then the function $f$ is continuous if and only if each function $f_\alpha$ is continuous.
\end{problem}

\begin{proof}
    In general $f_\alpha = \pi_\alpha \circ f$, in which by assumption of continuity of $f$ makes $f_\alpha$ also continuous, as it is the composition of two continuous functions.
    \footnote{This holds for both product and box.}

    In the converse statement, we will have the apply the properties of the product.
    \footnote{To show $f$ is continuous, we can show that the preimages of subbasic sets are open.}
    Let $\pi_\beta^{-1}(V_\beta)$ be a general subbasis
    \footnote{The product topology is generated by subbasis
        \begin{equation*}
            \bigcup_{\alpha \in J} \{ \pi_\beta^{-1}(U_\alpha) : U_\alpha \text{ is open in } X_\alpha \}.
        \end{equation*}
    }
    element in $X$,
    where $V_\beta$ is open in $X_\beta$.
    Like the first implication, the preimage is also open since
    $f^{-1}(\pi_\beta^{-1}(V_\beta)) = f_\beta^{-1}(V_\beta)$, which is open.
\end{proof}

\section{Quotients}

\begin{problem}[\S22 Ex. 2]
$ $
\begin{itemize}
    \item[(a)]
        Let $p: X \to Y$ be a continuous map.
        Show that if there is a continuous map $f : Y \to X$ such that $p \circ f$ equals the identity map of $Y$, then $p$ is a quotient map.
        \footnote{$p$ is a \textbf{quotient map} when $V$ open $\Leftrightarrow$ $p^{-1}(V)$ open; $p$ also needs to be surjective.
        A quotient map that is bijective is just a homeomorphic map.}
        \footnote{We can give $Y$ the \textbf{quotient topology} which is the finest topology that makes $p$ a quotient map.}

    \item[(b)]
        If $A \subseteq X$, a retraction of $X$ onto $A$ is a continuous map $r : X \to A$ such that $r(a) = a$ for each $a \in A$.
        Show that a retraction is a quotient map.
\end{itemize}
\end{problem}

Since $p$ is continuous by assumption, we only need to show one side of the implication.
Let $A = p^{-1}(B)$ be an open set in $X$.
\footnote{These are precisely the \textbf{saturated} open sets of $X$.}
Since $f$ is continuous, $f^{-1}(A)$ is open.
Using the following equality, we also get that $B$ is open.
\begin{equation*}
	f^{-1}(A) = f^{-1}(p^{-1}(B)) = (p \circ f)^{-1}(B) = B
\end{equation*}

Consider $f: A \to X$ given by $f(a) = a$.
$r \circ f$ is the identity map on $A$, hence by applying (a), we  get that $r$ is a quotient map.

\begin{problem}[\S 22 Ex. 4]
    $ $
    \begin{itemize}
        \item[(a)]
            Define an equivalence relation
            \footnote{This \textbf{equivalence relation} defines an \textbf{equivalence class} of $[x_0 \times y_0]$, all $x_1 \times y_1$ satisfy this equation i.e. parabolas.}
            on the plane $X = \mathbb{R}^2$ as follows.
            \begin{equation*}
                x_0 \times y_0 \sim x_1 \times y_1 \text{ if } x_0 + y_0^2 = x_1 + y_1^2
            \end{equation*}
            Let $X^*$ be the corresponding quotient space.
            \footnote{The open sets in the \textbf{quotient space} are collections of classes such that the decomposition in the original space is open.}
            It is homeomorphic to a familiar space; what is it?
            Hint: Set $g(x \times y) = x + y^2$.

        \item[(b)]
            Repeat (a) for the equivalence relation
            \begin{equation*}
                x_0 \times y_0 \sim x_1 \times y_1 \text{ if } x_0^2 + y_0^2 = x_1^2 + y_1^2
            \end{equation*}
    \end{itemize}
\end{problem}

\begin{proof}
    $X^*$ is the set of parabolas defined by the equation $x + y^2 = 0$ where the general element is the equivalence class of $[x \times 0]$.
    \footnote{
            \centering
            \begin{xy}
                (0,20)*+{\mathbb{R}^2}="a"; 
                (20,20)*+{\mathbb{R}}="b";
                (0, 0)*+{X^*}="c";
                {\ar "a";"b"}?*!/_8pt/{g};
                {\ar "a";"c"}?*!/^6pt/{p};
                {\ar "c";"b"}?*!/^8pt/{f};
            \end{xy}
    }

    Now consider the given function $g$, it suffices to show that $g$ is a quotient map, since that would imply a homeomorphism between $X^*$ and $\mathbb{R}$.
    \footnote{The whole point is that $g$ is not injective, but if we can `factor' into a quotient space (the parabolas) that is in bijection with the reals, then $g$ being quotient is equivalent to the space of parabolas is homeomorphic to the real line.}
    $g$ is clearly continuous and surjective, and we can take the right-inverse $x \mapsto x \times 0$, which indeeds gives the identity map of $\mathbb{R}$.
    By \S 22 Ex. 2, $g$ is a quotient, as needed.
\end{proof}

\end{document}
