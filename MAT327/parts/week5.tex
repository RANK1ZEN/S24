\documentclass[../../main.tex]{subfiles}

\begin{document}

\section{Week V - Metric topologies}

\begin{problem}[4]
    Let $X$ and $Y$ be topological spaces, and let $f : X \to Y$ be a function.
    \begin{itemize}
        \item[(b)] Prove that if $X$ is sequential
            \footnote{$X$ is sequential $\Leftrightarrow$ for every $A \subseteq X$, and every $x \in \overline{A}$, there is a sequence $(x_n)$ in $A$ with $x_n \to x$.}
            and $f(x_n) \to f(x)$ whenever $x_n \to x$, then $f$ is continuous.
    \end{itemize}
\end{problem}

\begin{proof}[Proof of (a)]
    Let $V$ be a neighbourhood of $f(x)$.
    \footnote{We want to show there is a tail of $f(x_n)$ inside $V$.}
    By continuity of $f$, we have a neighbourhood around $x$ such that $f(U) \subseteq V$.
    By the limit of $(x_n)$, we have some large $N$ such that $x_n \in U$ for $n \ge N$.
    But since $x_n \in U$, we have $f(x_n) \in V$, and this shows $f(x_n) \to f(x)$.
\end{proof}

\begin{proof}[Proof of (b)]
    Suppose $A$ is a subset of $X$.
    \footnote{We want to show $f(\overline{A}) \subseteq \overline{f(A)}$.}
    Fix $x \in \overline{A}$, since $X$ is sequential, we have a sequence $(x_n)$ in $A$ with $x_n \to x$.
    By assumption this is equivalent to $f(x_n) \to f(x)$.
    This is equivalent to $f(x)$ is a limit point and thus $f(x) \in \overline{f(A)}$.
    FIXME $f(\overline{A}) \subseteq \overline{f(A)}$.
\end{proof}

\begin{problem}[5]
    Prove that the lexicographic plane is metrizable.
\end{problem}
\begin{proof}
    \footnote{Find a metric on $\mathbb{R}^2$ such that the topology generated is the lexicographic topology.}
    Take the metric
    \[
        d(x_1 \times y_1, x_2 \times y_2) = \begin{cases}
            \min\{|y_1 - y_2|, 1\} &, x_1 = x_2 \\
            1 &, x_1 \ne x_2
        \end{cases}
    \]
    where $x_1 \times y_1, x_2 \times y_2 \in \mathbb{R} \times \mathbb{R}$.
    First we will show this is a metric.
    \footnote{Show $d(x, y) \ge 0$, $d(x, y) = d(y, x)$, and the triangle inequality.}
    Clearly, the postive and symmetric properties hold.
    To show the triangle inequality, suppose three points $x, y, z$.
    We have $3$ cases to check.
    Case 1, every point has a different first coordinate, this case holds.
    Case 2, 2 points have the same first coordinate, 
    This is indeed a metric.

    Now we can compare the basis of the two topologies.
    The basis of the metric topology are open sets in the lexicographic topology, and at the same time, the basis for lexicographic topology is in the metric topology.
    Thus we are done.
\end{proof}

\end{document}
