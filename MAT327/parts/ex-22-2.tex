\documentclass[../../main.tex]{subfiles}

\begin{document}

\begin{problem}[\S22 Ex. 2]
$ $
\begin{itemize}
    \item[(a)] Let $p: X \to Y$ be a continuous map.
	      Show that if there is a continuous map $f : Y \to X$ such that $p \circ f$ equals the identity map of $Y$, then $p$ is a quotient map.
          \footnote{This basically says that if you can find a right-inverse, its a quotient map.}

    \item[(b)] If $A \subseteq X$, a retraction of $X$ onto $A$ is a continuous map $r : X \to A$ such that $r(a) = a$ for each $a \in A$.
	      Show that a retraction is a quotient map.
\end{itemize}
\end{problem}

Since $p$ is continuous by assumption, we only need to show one side of the implication.
\footnote{Show that $p$ maps saturated open sets of $X$ to open sets of $Y$.}
Let $A = p^{-1}(B)$ be an open set in $X$.
Since $f$ is continuous, $f^{-1}(A)$ is open.
\footnote{For an open set in the codomain, the preimage is also open.}
Using the following equality,
\begin{equation*}
	f^{-1}(A) = f^{-1}(p^{-1}(B)) = (p \circ f)^{-1}(B) = B,
\end{equation*}
we also get that $B$ is open.

Consider $f: A \to X$ given by $f(a) = a$.
\footnote{Also called the inclusion map.}
$r \circ f$ is the identity map on $A$, hence by applying (a), we  get that $r$ is a quotient map.

\end{document}
